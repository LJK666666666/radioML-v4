\documentclass[aspectratio=169]{beamer}
\usepackage[UTF8]{ctex}
\usepackage{graphicx}
\usepackage{booktabs}
\usepackage{amsmath}
\usepackage{xcolor}
\usepackage{tikz}
\usepackage{multicol}

% 主题设置
\usetheme{Madrid}
\usecolortheme{default}

% 自定义颜色
\definecolor{myblue}{RGB}{0,102,204}
\definecolor{mygreen}{RGB}{0,153,76}
\definecolor{myred}{RGB}{204,0,51}

% 设置主题颜色
\setbeamercolor{structure}{fg=myblue}
\setbeamercolor{frametitle}{bg=myblue,fg=white}

% 标题页信息
\title[基于改进YOLOv5的反光衣检测研究]{基于改进YOLOv5的反光衣检测研究:\\GhostNet、CA注意力与WIoU损失的集成与评估}
\subtitle{工业安全检测中的深度学习应用}
\author{研究团队}
\institute{计算机科学与技术学院}
\date{\today}

% 去掉导航符号
\setbeamertemplate{navigation symbols}{}

% 页脚设置
\setbeamertemplate{footline}[frame number]

\begin{document}

% 标题页
\begin{frame}
    \titlepage
\end{frame}

% 目录页
\begin{frame}{目录}
    \tableofcontents
\end{frame}

% 第一部分:研究背景
\section{研究背景与动机}

\begin{frame}{工业安全检测的重要性}
    \begin{columns}
        \begin{column}{0.6\textwidth}
            \begin{itemize}
                \item \textbf{安全事故频发}:工业现场安全事故造成巨大损失
                \item \textbf{人工监控局限}:传统人工监控效率低、成本高
                \item \textbf{智能化需求}:急需智能化安全监控系统
                \item \textbf{实时性要求}:毫秒级响应,及时预警
            \end{itemize}
        \end{column}
        \begin{column}{0.4\textwidth}
            \begin{center}
                \includegraphics[width=\textwidth]{figure/safety_score_comparison.png}
            \end{center}
        \end{column}
    \end{columns}
    
    \vspace{0.5cm}
    \begin{alertblock}{核心挑战}
        如何在保证检测精度的同时,实现模型的轻量化和实时性?
    \end{alertblock}
\end{frame}

\begin{frame}{反光衣检测任务特点}
    \begin{columns}
        \begin{column}{0.5\textwidth}
            \textbf{任务特殊性:}
            \begin{itemize}
                \item 类别不平衡问题严重
                \item 环境复杂(光照、遮挡)
                \item 误报代价不对称
                \item 实时性要求严格
            \end{itemize}
        \end{column}
        \begin{column}{0.5\textwidth}
            \textbf{技术挑战:}
            \begin{itemize}
                \item 小目标检测困难
                \item 反光材质识别复杂
                \item 模型轻量化需求
                \item 边缘设备部署限制
            \end{itemize}
        \end{column}
    \end{columns}
    
    \vspace{0.5cm}
    \begin{block}{研究目标}
        构建高精度、轻量化的反光衣检测模型,满足工业现场实时监控需求
    \end{block}
\end{frame}

% 第二部分:技术方案
\section{技术方案与创新点}

\begin{frame}{YOLOv5基础架构}
    \begin{center}
        \includegraphics[width=0.9\textwidth]{figure/yolov5s_architecture.png}
    \end{center}
    
    \begin{itemize}
        \item \textbf{Backbone}:特征提取网络,采用CSP结构
        \item \textbf{Neck}:特征融合网络,FPN+PAN设计
        \item \textbf{Head}:检测头,多尺度目标检测
    \end{itemize}
\end{frame}

\begin{frame}{改进策略一:GhostNet轻量化}
    \begin{center}
        \includegraphics[width=0.95\textwidth]{figure/ghostnet_module.png}
    \end{center}
    
    \begin{columns}
        \begin{column}{0.5\textwidth}
            \textbf{核心思想:}
            \begin{itemize}
                \item 特征图冗余性利用
                \item 廉价操作生成Ghost特征
                \item 大幅减少参数量和计算量
            \end{itemize}
        \end{column}
        \begin{column}{0.5\textwidth}
            \textbf{技术优势:}
            \begin{itemize}
                \item 参数量减少27.6\%
                \item GFLOPs降低34.2\%
                \item 保持相对较高精度
            \end{itemize}
        \end{column}
    \end{columns}
\end{frame}

\begin{frame}{改进策略二:CA注意力机制}
    \begin{center}
        \includegraphics[width=0.8\textwidth]{figure/ca_attention.png}
    \end{center}
    
    \begin{itemize}
        \item \textbf{位置感知}:同时捕获水平和垂直方向的位置信息
        \item \textbf{轻量设计}:避免大量计算开销
        \item \textbf{精确定位}:增强模型对目标位置的感知能力
    \end{itemize}
\end{frame}

\begin{frame}{改进策略三:WIoU损失函数}
    \begin{columns}
        \begin{column}{0.6\textwidth}
            \textbf{传统IoU损失的问题:}
            \begin{itemize}
                \item 梯度消失问题
                \item 收敛速度慢
                \item 对质量差的样本敏感
            \end{itemize}
            
            \vspace{0.5cm}
            \textbf{WIoU的改进:}
            \begin{itemize}
                \item 动态聚焦机制
                \item 智能梯度分配
                \item 提升收敛效率
            \end{itemize}
        \end{column}
        \begin{column}{0.4\textwidth}
            \begin{block}{核心公式}
                \begin{align*}
                    L_{WIoU} &= r_{WIoU} \times (1-IoU) \times \beta \\
                    r_{WIoU} &= \exp\left(\frac{d^2}{c_w^2 + c_h^2}\right) \\
                    \beta &= \left(\frac{IoU}{\alpha}\right)^{\gamma}
                \end{align*}
            \end{block}
        \end{column}
    \end{columns}
\end{frame}

% 第三部分:实验设计
\section{实验设计与数据集}

\begin{frame}{实验设计思路}
    \begin{center}
        \begin{tikzpicture}[node distance=1.5cm, every node/.style={text width=2cm, align=center}]
            \node[draw, rectangle, fill=myblue!20] (baseline) {基线模型\\yolov5s\_};
            \node[draw, rectangle, fill=mygreen!20, right of=baseline] (ghost) {Ghost轻量化\\yolov5s-ghost\_1\_};
            \node[draw, rectangle, fill=mygreen!20, below of=baseline] (ca) {CA注意力\\yolov5s-ghost\_2\_};
            \node[draw, rectangle, fill=mygreen!20, below of=ghost] (wiou) {WIoU损失\\yolov5s-ghost\_3\_};
            \node[draw, rectangle, fill=myred!20, below right of=ca] (combined) {组合模型\\yolov5s-ghost\_123\_};

            \draw[->] (baseline) -- (ghost);
            \draw[->] (baseline) -- (ca);
            \draw[->] (baseline) -- (wiou);
            \draw[->] (ghost) -- (combined);
            \draw[->] (ca) -- (combined);
            \draw[->] (wiou) -- (combined);
        \end{tikzpicture}
    \end{center}
    
    \textbf{消融实验设计:}系统性评估每个改进模块的独立贡献和组合效应
\end{frame}

\begin{frame}{数据集与评价指标}
    \begin{columns}
        \begin{column}{0.5\textwidth}
            \textbf{SafetyVests数据集:}
            \begin{itemize}
                \item 总计3,897张图像
                \item 训练集:2,728张(70\%)
                \item 验证集:779张(20\%)
                \item 测试集:390张(10\%)
                \item 两个类别:Safety Vest、No-Safety Vest
            \end{itemize}
        \end{column}
        \begin{column}{0.5\textwidth}
            \textbf{评价指标体系:}
            \begin{itemize}
                \item \textbf{精度指标}:Precision、Recall、mAP
                \item \textbf{效率指标}:参数量、GFLOPs、FPS
                \item \textbf{安全评分}:$S = \alpha \times R_{critical} + (1-\alpha) \times mAP$
            \end{itemize}
        \end{column}
    \end{columns}
    
    \begin{alertblock}{安全评分说明}
        $\alpha = 0.6$,强调对关键类别(No-Safety Vest)召回率的重视
    \end{alertblock}
\end{frame}

% 第四部分:实验结果
\section{实验结果与分析}

\begin{frame}{总体性能对比}
    \begin{center}
        \includegraphics[width=0.85\textwidth]{figure/map_comparison.png}
    \end{center}
    
    \textbf{关键发现:}基线模型在mAP@0.5和mAP@0.5:0.95上均表现最佳
\end{frame}

\begin{frame}{关键类别性能分析}
    \begin{center}
        \includegraphics[width=0.85\textwidth]{figure/critical_class_metrics.png}
    \end{center}
    
    \begin{itemize}
        \item \textbf{No-Safety Vest类别}:安全关键类别,漏报代价极高
        \item \textbf{基线模型优势}:在Precision、Recall和mAP上均领先
        \item \textbf{CA注意力效果}:在该类别上有一定提升
    \end{itemize}
\end{frame}

\begin{frame}{模型复杂度与效率对比}
    \begin{center}
        \includegraphics[width=0.9\textwidth]{figure/complexity_vs_efficiency.png}
    \end{center}
\end{frame}

\begin{frame}{安全评分综合评估}
    \begin{center}
        \includegraphics[width=0.85\textwidth]{figure/safety_score_comparison.png}
    \end{center}
    
    \begin{block}{评分结果}
        \begin{itemize}
            \item \textbf{yolov5s\_}:0.7334(最高)
            \item \textbf{yolov5s-ghost\_3\_}:0.7208(第二)
            \item \textbf{yolov5s-ghost\_2\_}:0.7100(第三)
        \end{itemize}
    \end{block}
\end{frame}

% 第五部分:深入分析
\section{深入分析与讨论}

\begin{frame}{意外发现:改进策略的局限性}
    \begin{alertblock}{核心发现}
        理论上先进的改进策略在特定任务上可能适得其反
    \end{alertblock}
    
    \begin{columns}
        \begin{column}{0.5\textwidth}
            \textbf{GhostNet的双刃剑:}
            \begin{itemize}
                \item 参数量减少27.6\%
                \item 但mAP下降7.0\%
                \item 特征压缩损失关键信息
            \end{itemize}
        \end{column}
        \begin{column}{0.5\textwidth}
            \textbf{CA注意力的开销:}
            \begin{itemize}
                \item 推理时间增加72.7\%
                \item FPS下降41.9\%
                \item 精度提升微乎其微
            \end{itemize}
        \end{column}
    \end{columns}
    
    \vspace{0.5cm}
    \textbf{组合效应:}三个模块同时使用时,性能下降18.3\%(负协同效应)
\end{frame}

\begin{frame}{深层原因分析}
    \begin{enumerate}
        \item \textbf{任务特异性}
        \begin{itemize}
            \item 反光衣检测需要精细纹理特征
            \item GhostNet的特征压缩可能丢失关键信息
            \item 通用改进策略未必适合特定垂直领域
        \end{itemize}
        
        \item \textbf{数据集规模限制}
        \begin{itemize}
            \item 相对较小的数据集(1,330张)
            \item 复杂模型容易过拟合
            \item 简单模型反而更稳定
        \end{itemize}
        
        \item \textbf{模块间干扰}
        \begin{itemize}
            \item 不同改进策略可能相互冲突
            \item 特征分布不匹配
            \item 训练难度增加
        \end{itemize}
    \end{enumerate}
\end{frame}

\begin{frame}{方法论贡献}
    \begin{block}{科学价值}
        \textbf{"负面"结果的重要意义}:证明了盲目追求新技术的风险
    \end{block}
    
    \begin{columns}
        \begin{column}{0.5\textwidth}
            \textbf{评估框架:}
            \begin{itemize}
                \item 任务驱动的评价体系
                \item 安全评分概念
                \item 系统性消融实验
            \end{itemize}
        \end{column}
        \begin{column}{0.5\textwidth}
            \textbf{实践指导:}
            \begin{itemize}
                \item 模型选型科学依据
                \item 避免过度工程化
                \item 重视基线模型价值
            \end{itemize}
        \end{column}
    \end{columns}
    
    \begin{alertblock}{核心启示}
        在安全关键应用中,稳定可靠的基线模型可能比复杂的改进模型更有价值
    \end{alertblock}
\end{frame}

% 第六部分:总结与展望
\section{总结与展望}

\begin{frame}{主要贡献与结论}
    \begin{enumerate}
        \item \textbf{系统性评估}
        \begin{itemize}
            \item 完成了GhostNet、CA注意力、WIoU损失的全面评估
            \item 揭示了改进策略在特定任务上的局限性
        \end{itemize}
        
        \item \textbf{方法论创新}
        \begin{itemize}
            \item 提出了安全评分评价体系
            \item 建立了任务驱动的模型评估框架
        \end{itemize}
        
        \item \textbf{实践价值}
        \begin{itemize}
            \item 为工业安全检测提供了模型选型指导
            \item 证明了基线模型在特定场景下的优势
        \end{itemize}
    \end{enumerate}
    
    \begin{block}{最终推荐}
        对于反光衣检测任务,推荐使用原始YOLOv5s模型
    \end{block}
\end{frame}

\begin{frame}{未来研究方向}
    \begin{columns}
        \begin{column}{0.5\textwidth}
            \textbf{技术改进:}
            \begin{itemize}
                \item 探索其他轻量化方案
                \item 设计任务特定的损失函数
                \item 优化后处理策略
                \item 多模态信息融合
            \end{itemize}
        \end{column}
        \begin{column}{0.5\textwidth}
            \textbf{应用拓展:}
            \begin{itemize}
                \item 边缘计算部署优化
                \item 其他安全装备检测
                \item 持续学习机制
                \item 可解释性增强
            \end{itemize}
        \end{column}
    \end{columns}
    
    \vspace{0.5cm}
    \begin{block}{长远目标}
        构建通用的工业安全智能监控平台,实现多场景、多任务的统一检测
    \end{block}
\end{frame}

\begin{frame}
    \begin{center}
        \Huge 谢谢聆听!

        \vspace{1cm}
        \Large 欢迎提问与讨论
    \end{center}
\end{frame}

\end{document}
