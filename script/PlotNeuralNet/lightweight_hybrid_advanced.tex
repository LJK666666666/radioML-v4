% Lightweight Hybrid Model - PlotNeuralNet LaTeX Code
% 轻量级混合模型的高质量3D可视化
% 使用方法: 1) 下载PlotNeuralNet: https://github.com/HarisIqbal88/PlotNeuralNet
%          2) 将此文件放入PlotNeuralNet目录
%          3) 运行: bash png_latex.sh lightweight_hybrid_advanced

\documentclass[border=8pt, multi, tikz]{standalone}
\usepackage{import}
\subimport{./layers/}{init}
\usetikzlibrary{positioning}
\usetikzlibrary{3d}

% 颜色定义
\def\ConvColor{rgb:yellow,5;red,2.5;white,5}
\def\ConvReluColor{rgb:yellow,5;red,5;white,5}
\def\PoolColor{rgb:red,1;black,0.3}
\def\DenseColor{rgb:blue,5;red,2.5;white,5}
\def\ComplexColor{rgb:green,5;blue,2.5;white,3}
\def\ResColor{rgb:purple,5;blue,2;white,3}

\begin{document}
\begin{tikzpicture}
\tikzstyle{connection}=[ultra thick,every node/.style={sloped,allow upside down},draw=\edgecolor,opacity=0.6]

% 输入层
\pic[shift={(0,0,0)}] at (0,0,0) {Box={
    name=input,
    caption=I/Q Signal\\(2×128),
    fill=\ConvColor,
    height=20,
    width=2,
    depth=64
}};

% 预处理
\pic[shift={(1.5,0,0)}] at (0,0,0) {Box={
    name=permute,
    caption=Permute\\(128×2),
    fill=\ConvColor,
    height=64,
    width=2,
    depth=20
}};

% 复数特征提取阶段
\pic[shift={(3,0,0)}] at (0,0,0) {Box={
    name=conv1,
    caption=ComplexConv1D\\filters=32\\kernel=5,
    fill=\ComplexColor,
    height=64,
    width=8,
    depth=18
}};

\pic[shift={(4.2,0,0)}] at (0,0,0) {Box={
    name=bn1,
    caption=Complex\\BatchNorm,
    fill=\ComplexColor,
    height=64,
    width=8,
    depth=15
}};

\pic[shift={(5.4,0,0)}] at (0,0,0) {Box={
    name=act1,
    caption=Complex\\CReLU,
    fill=\ComplexColor,
    height=64,
    width=8,
    depth=12
}};

\pic[shift={(6.6,0,0)}] at (0,0,0) {Box={
    name=pool1,
    caption=Complex\\MaxPool1D,
    fill=\PoolColor,
    height=32,
    width=8,
    depth=10
}};

% 残差学习阶段
\pic[shift={(8.2,0,0)}] at (0,0,0) {Box={
    name=res1,
    caption=ResBlock-1\\filters=64\\skip connection,
    fill=\ResColor,
    height=32,
    width=12,
    depth=9
}};

\pic[shift={(10,0,0)}] at (0,0,0) {Box={
    name=res2,
    caption=ResBlock-2\\filters=128\\skip connection,
    fill=\ResColor,
    height=16,
    width=16,
    depth=8
}};

\pic[shift={(12,0,0)}] at (0,0,0) {Box={
    name=res3,
    caption=ResBlock-3\\filters=256\\skip connection,
    fill=\ResColor,
    height=8,
    width=20,
    depth=7
}};

% 全局特征处理
\pic[shift={(14.5,0,0)}] at (0,0,0) {Box={
    name=globalpool,
    caption=Complex\\GlobalAvgPool,
    fill=\PoolColor,
    height=4,
    width=20,
    depth=6
}};

\pic[shift={(16.5,0,0)}] at (0,0,0) {Box={
    name=dense1,
    caption=Complex\\Dense 512,
    fill=\ComplexColor,
    height=3,
    width=25,
    depth=5
}};

% 实数分类阶段
\pic[shift={(18.8,0,0)}] at (0,0,0) {Box={
    name=magnitude,
    caption=Complex\\Magnitude,
    fill=\DenseColor,
    height=2,
    width=25,
    depth=4
}};

\pic[shift={(21,0,0)}] at (0,0,0) {Box={
    name=dense2,
    caption=Dense 256\\+ ReLU,
    fill=\DenseColor,
    height=2,
    width=18,
    depth=3
}};

\pic[shift={(23,0,0)}] at (0,0,0) {Box={
    name=dropout,
    caption=Dropout\\rate=0.3,
    fill=\DenseColor,
    height=2,
    width=18,
    depth=2
}};

\pic[shift={(25,0,0)}] at (0,0,0) {Box={
    name=output,
    caption=Output\\11 classes\\Softmax,
    fill=\DenseColor,
    height=1,
    width=6,
    depth=1
}};

% 连接线
\draw [connection] (input-east) -- node {\midarrow} (permute-west);
\draw [connection] (permute-east) -- node {\midarrow} (conv1-west);
\draw [connection] (conv1-east) -- node {\midarrow} (bn1-west);
\draw [connection] (bn1-east) -- node {\midarrow} (act1-west);
\draw [connection] (act1-east) -- node {\midarrow} (pool1-west);
\draw [connection] (pool1-east) -- node {\midarrow} (res1-west);
\draw [connection] (res1-east) -- node {\midarrow} (res2-west);
\draw [connection] (res2-east) -- node {\midarrow} (res3-west);
\draw [connection] (res3-east) -- node {\midarrow} (globalpool-west);
\draw [connection] (globalpool-east) -- node {\midarrow} (dense1-west);
\draw [connection] (dense1-east) -- node {\midarrow} (magnitude-west);
\draw [connection] (magnitude-east) -- node {\midarrow} (dense2-west);
\draw [connection] (dense2-east) -- node {\midarrow} (dropout-west);
\draw [connection] (dropout-east) -- node {\midarrow} (output-west);

% 跳跃连接(残差连接)
\draw [connection, dashed, color=gray] (res1-north) to [out=90,in=90] (res2-north);
\draw [connection, dashed, color=gray] (res2-north) to [out=90,in=90] (res3-north);

% 标题和注释
\node[above=1cm of input] {\Large\textbf{Lightweight Hybrid Model for RF Signal Classification}};
\node[below=0.5cm of output] {\small Parameters: ~1.3M | Accuracy: 65.38\% | Inference: ~2.3ms};

\end{tikzpicture}
\end{document}
