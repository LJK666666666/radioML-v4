\documentclass[aspectratio=169]{beamer}
\usetheme{Madrid}
\usecolortheme{default}

% 中文支持
\usepackage{ctex}
\usepackage{amsmath}
\usepackage{graphicx}
\usepackage{booktabs}
\usepackage{algorithm}
\usepackage{algorithmic}

% 标题页信息
\title{RML2016.10a数据集的分类预测任务研究}
\subtitle{基于改进ResNet的无线电调制识别}
\author{您的姓名}
\institute{您的机构}
\date{\today}

\begin{document}

% 标题页
\begin{frame}
\titlepage
\end{frame}

% 目录
\begin{frame}{目录}
\tableofcontents
\end{frame}

% 第一部分:研究背景
\section{研究背景与目标}

\begin{frame}{研究背景}
\begin{itemize}
    \item 无线电调制识别是现代通信系统的重要组成部分
    \item RML2016.10a数据集是该领域的经典基准数据集
    \item 现有方法在低信噪比条件下性能有限
    \item 需要更鲁棒、高效的调制识别算法
\end{itemize}
\end{frame}

\begin{frame}{研究目标}
\begin{itemize}
    \item 在RML2016.10a数据集上实现高精度调制识别
    \item 探索多种神经网络架构的性能
    \item 通过创新技术提升分类准确率
    \item 超越现有SOTA方法的性能指标
\end{itemize}
\end{frame}

% 第二部分:方法论
\section{方法论}

\begin{frame}{数据集划分策略}
\begin{itemize}
    \item \textbf{测试集比例}: 0.2(与多数论文保持一致)
    \item \textbf{训练集比例}: 0.72
    \item \textbf{验证集比例}: 0.08
    \item 经过实验对比,该划分比例能更好地训练神经网络
\end{itemize}

\vspace{1em}
\begin{center}
\begin{tabular}{|c|c|c|}
\hline
训练集 & 验证集 & 测试集 \\
\hline
72\% & 8\% & 20\% \\
\hline
\end{tabular}
\end{center}
\end{frame}

\begin{frame}{网络架构对比实验}
尝试了多种神经网络架构:

\begin{itemize}
    \item \textbf{FCNN} - 全连接神经网络
    \item \textbf{CNN1D} - 一维卷积神经网络
    \item \textbf{CNN2D} - 二维卷积神经网络
    \item \textbf{ResNet} - 残差神经网络
    \item \textbf{Transformer} - 注意力机制网络
    \item \textbf{ComplexCNN} - 复数卷积神经网络
\end{itemize}

\vspace{0.5em}
\textcolor{red}{\textbf{实验结果:ResNet效果最佳}}
\end{frame}

% 第三部分:创新点
\section{技术创新}

\begin{frame}{创新点一:高斯过程回归去噪}
\textbf{理论基础:}
\begin{itemize}
    \item 噪声主要是加性高斯白噪声
    \item 每点处信号服从高斯分布且相互独立
    \item 信号可视作高斯过程
\end{itemize}

\textbf{实现方法:}
\begin{itemize}
    \item 根据I/Q通道数据和信噪比计算噪声标准差
    \item 设置自适应的length\_scale参数
    \item 采用高斯过程回归进行信号去噪
\end{itemize}

\textbf{效果:}\textcolor{green}{分类准确率显著提升}
\end{frame}

\begin{frame}{创新点二:旋转数据增强}
\textbf{灵感来源:}
\begin{itemize}
    \item 观察星座图发现信号的旋转属性
    \item 借鉴ULCNN论文的旋转数据增强思想
\end{itemize}

\textbf{实现方法:}
\begin{itemize}
    \item 在训练集上对信号进行旋转变换
    \item 旋转角度:90°、180°、270°
    \item 扩充训练数据量,增强模型鲁棒性
\end{itemize}
\end{frame}

\begin{frame}{创新点三:混合神经网络架构}
\textbf{观察发现:}
\begin{itemize}
    \item ComplexCNN收敛速度快
    \item 准确率仅次于ResNet,优于CNN1D和CNN2D
\end{itemize}

\textbf{融合策略:}
\begin{itemize}
    \item 将ResNet的输入层改为复数层
    \item 直接对复数信号进行处理
    \item 得到ResNet和ComplexCNN的混合架构
\end{itemize}

\textbf{优势:}结合了两种架构的优点
\end{frame}

% 第四部分:实验结果
\section{实验结果}

\begin{frame}{基线对比}
\textbf{收集了5篇相关论文作为baseline}

\begin{center}
\begin{tabular}{|l|c|}
\hline
\textbf{方法} & \textbf{准确率} \\
\hline
Previous SOTA & < 65.38\% \\
\hline
本研究(改进前) & - \\
\hline
\textcolor{red}{\textbf{本研究(最终)}} & \textcolor{red}{\textbf{65.38\%}} \\
\hline
\end{tabular}
\end{center}

\vspace{1em}
\textcolor{green}{\textbf{成功超越了Previous SOTA的分类准确率指标!}}
\end{frame}

\begin{frame}{改进效果分析}
\textbf{三次关键改进:}

\begin{enumerate}
    \item \textbf{高斯过程回归去噪} $\rightarrow$ 显著提升准确率
    \item \textbf{旋转数据增强} $\rightarrow$ 增强模型鲁棒性
    \item \textbf{混合网络架构} $\rightarrow$ 融合优势特征
\end{enumerate}

\vspace{1em}
\textbf{最终结果:}
\begin{itemize}
    \item 分类准确率达到 \textcolor{red}{\textbf{65.38\%}}
    \item 超越现有SOTA方法
    \item 验证了创新技术的有效性
\end{itemize}
\end{frame}

% 第五部分:总结与展望
\section{总结与展望}

\begin{frame}{主要贡献}
\begin{itemize}
    \item \textbf{系统性对比}:全面评估了多种神经网络架构
    \item \textbf{创新去噪}:提出基于高斯过程回归的自适应去噪方法
    \item \textbf{数据增强}:应用旋转变换增强训练数据
    \item \textbf{架构融合}:设计ResNet-ComplexCNN混合网络
    \item \textbf{性能突破}:实现了SOTA性能的超越
\end{itemize}
\end{frame}

\begin{frame}{未来工作}
\begin{itemize}
    \item 探索更多数据增强技术
    \item 研究注意力机制在调制识别中的应用
    \item 优化模型复杂度,提升推理效率
    \item 扩展到更多调制类型和数据集
    \item 研究实时处理的硬件加速方案
\end{itemize}
\end{frame}

% 感谢页面
\begin{frame}
\begin{center}
{\Huge 谢谢!}

\vspace{2em}
{\Large 欢迎提问与讨论}

\vspace{2em}
\texttt{联系方式:your.email@example.com}
\end{center}
\end{frame}

\end{document}
