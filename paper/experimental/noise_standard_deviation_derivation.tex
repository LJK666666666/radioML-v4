\documentclass[12pt]{article}
\usepackage[utf8]{inputenc}
\usepackage{amsmath}
\usepackage{amsfonts}
\usepackage{amssymb}
\usepackage{mathtools}
\usepackage{geometry}
\usepackage{xeCJK}

\geometry{a4paper, margin=1in}

\title{噪声标准差的数学推导过程}
\author{}
\date{}

\begin{document}

\maketitle

\section{问题描述}

在无线通信系统中,接收信号通常受到加性高斯白噪声(AWGN)的干扰。为了在高斯过程回归(GPR)去噪过程中准确设置噪声水平参数,需要根据已知的信噪比(SNR)和接收信号功率计算噪声的标准差。本文详细推导这一计算过程。

\section{基本定义与假设}

\subsection{信号模型}

设原始无噪基带信号为复数形式:
\begin{equation}
s[n] = s_I[n] + js_Q[n]
\end{equation}
其中$s_I[n]$和$s_Q[n]$分别为同相和正交分量,$j$为虚数单位。

接收信号受到加性复高斯白噪声的干扰:
\begin{equation}
r[n] = s[n] + w[n]
\end{equation}
其中$w[n] = w_I[n] + jw_Q[n]$为复高斯白噪声。

\subsection{噪声特性假设}

对于复高斯白噪声$w[n]$,其同相分量$w_I[n]$和正交分量$w_Q[n]$具有以下特性:
\begin{itemize}
\item 相互独立:$w_I[n] \perp w_Q[n]$
\item 零均值:$\mathbb{E}[w_I[n]] = \mathbb{E}[w_Q[n]] = 0$
\item 等方差:$\text{Var}(w_I[n]) = \text{Var}(w_Q[n]) = \sigma_n^2$
\item 高斯分布:$w_I[n], w_Q[n] \sim \mathcal{N}(0, \sigma_n^2)$
\end{itemize}

因此,复高斯白噪声可表示为:
\begin{equation}
w[n] \sim \mathcal{CN}(0, \sigma_n^2)
\end{equation}

\section{功率关系推导}

\subsection{功率定义}

原始信号功率定义为:
\begin{equation}
P_s = \mathbb{E}[|s[n]|^2] = \mathbb{E}[s_I[n]^2 + s_Q[n]^2]
\end{equation}

噪声功率定义为:
\begin{equation}
P_w = \mathbb{E}[|w[n]|^2] = \mathbb{E}[w_I[n]^2 + w_Q[n]^2]
\end{equation}

接收信号功率定义为:
\begin{equation}
P_r = \mathbb{E}[|r[n]|^2] = \mathbb{E}[r_I[n]^2 + r_Q[n]^2]
\end{equation}

\subsection{噪声功率与方差的关系推导}

根据噪声的统计特性,噪声功率可以展开为:
\begin{align}
P_w &= \mathbb{E}[|w[n]|^2] \\
&= \mathbb{E}[(w_I[n] + jw_Q[n])(w_I[n] - jw_Q[n])] \\
&= \mathbb{E}[w_I[n]^2 + w_Q[n]^2] \\
&= \mathbb{E}[w_I[n]^2] + \mathbb{E}[w_Q[n]^2]
\end{align}

由于$w_I[n] \sim \mathcal{N}(0, \sigma_n^2)$,有:
\begin{equation}
\mathbb{E}[w_I[n]^2] = \text{Var}(w_I[n]) + (\mathbb{E}[w_I[n]])^2 = \sigma_n^2 + 0^2 = \sigma_n^2
\end{equation}

同理,对于$w_Q[n] \sim \mathcal{N}(0, \sigma_n^2)$:
\begin{equation}
\mathbb{E}[w_Q[n]^2] = \sigma_n^2
\end{equation}

因此,噪声功率与分量方差的关系为:
\begin{equation}
P_w = \sigma_n^2 + \sigma_n^2 = 2\sigma_n^2
\end{equation}

由此可得单个分量的噪声方差:
\begin{equation}
\sigma_n^2 = \frac{P_w}{2}
\end{equation}

噪声标准差为:
\begin{equation}
\sigma_n = \sqrt{\frac{P_w}{2}}
\end{equation}

\section{基于SNR的噪声功率计算}

\subsection{信噪比定义}

信噪比(SNR)定义为信号功率与噪声功率的比值:
\begin{equation}
\text{SNR}_{\text{linear}} = \frac{P_s}{P_w}
\end{equation}

对应的分贝值为:
\begin{equation}
\text{SNR}_{\text{dB}} = 10\log_{10}(\text{SNR}_{\text{linear}}) = 10\log_{10}\left(\frac{P_s}{P_w}\right)
\end{equation}

由分贝定义可得线性SNR:
\begin{equation}
\text{SNR}_{\text{linear}} = 10^{\text{SNR}_{\text{dB}}/10}
\end{equation}

\subsection{总功率关系}

假设信号与噪声不相关,即$\mathbb{E}[s[n]w^*[n]] = 0$,则接收信号的总功率为:
\begin{align}
P_r &= \mathbb{E}[|r[n]|^2] \\
&= \mathbb{E}[|s[n] + w[n]|^2] \\
&= \mathbb{E}[(s[n] + w[n])(s[n] + w[n])^*] \\
&= \mathbb{E}[s[n]s^*[n] + s[n]w^*[n] + w[n]s^*[n] + w[n]w^*[n]] \\
&= \mathbb{E}[|s[n]|^2] + \mathbb{E}[s[n]w^*[n]] + \mathbb{E}[w[n]s^*[n]] + \mathbb{E}[|w[n]|^2] \\
&= P_s + 0 + 0 + P_w \\
&= P_s + P_w
\end{align}

\subsection{噪声功率的显式表达}

从SNR定义可得:
\begin{equation}
P_s = \text{SNR}_{\text{linear}} \cdot P_w
\end{equation}

将此关系代入总功率方程:
\begin{align}
P_r &= P_s + P_w \\
&= \text{SNR}_{\text{linear}} \cdot P_w + P_w \\
&= P_w(\text{SNR}_{\text{linear}} + 1)
\end{align}

解出噪声功率:
\begin{equation}
P_w = \frac{P_r}{\text{SNR}_{\text{linear}} + 1}
\end{equation}

将线性SNR表达式代入:
\begin{equation}
P_w = \frac{P_r}{10^{\text{SNR}_{\text{dB}}/10} + 1}
\end{equation}

\section{最终的噪声标准差公式}

将噪声功率表达式代入噪声标准差公式:
\begin{equation}
\sigma_n = \sqrt{\frac{P_w}{2}} = \sqrt{\frac{P_r}{2(10^{\text{SNR}_{\text{dB}}/10} + 1)}}
\end{equation}

\subsection{实际计算中的接收功率估计}

在实际应用中,接收信号功率$P_r$通过有限样本估计:
\begin{equation}
\hat{P}_r = \frac{1}{M}\sum_{k=0}^{M-1}|r[k]|^2 = \frac{1}{M}\sum_{k=0}^{M-1}(r_I[k]^2 + r_Q[k]^2)
\end{equation}

其中$M$为样本数量,$r[k] = r_I[k] + jr_Q[k]$为第$k$个接收样本。

因此,实际使用的噪声标准差估计为:
\begin{equation}
\hat{\sigma}_n = \sqrt{\frac{\hat{P}_r}{2(10^{\text{SNR}_{\text{dB}}/10} + 1)}}
\end{equation}

\section{应用于高斯过程回归}

在GPR模型中,噪声水平参数$\alpha$设置为单个分量的噪声方差:
\begin{equation}
\alpha = \sigma_n^2 = \frac{P_w}{2} = \frac{P_r}{2(10^{\text{SNR}_{\text{dB}}/10} + 1)}
\end{equation}

这个参数被加入到GPR的协方差矩阵对角线上,用于建模观测噪声:
\begin{equation}
K_{noise} = K(X,X) + \alpha I
\end{equation}

其中$K(X,X)$为核函数矩阵,$I$为单位矩阵。

\section{数值验证示例}

假设已知条件:
\begin{itemize}
\item 接收信号功率:$P_r = 1.0$
\item 信噪比:$\text{SNR}_{\text{dB}} = 0$ dB
\end{itemize}

计算过程:
\begin{align}
\text{SNR}_{\text{linear}} &= 10^{0/10} = 1 \\
P_w &= \frac{1.0}{1 + 1} = 0.5 \\
\sigma_n^2 &= \frac{0.5}{2} = 0.25 \\
\sigma_n &= \sqrt{0.25} = 0.5
\end{align}

验证:原始信号功率$P_s = \text{SNR}_{\text{linear}} \times P_w = 1 \times 0.5 = 0.5$,总功率$P_s + P_w = 0.5 + 0.5 = 1.0 = P_r$ ✓

\section{结论}

通过严格的数学推导,我们得到了噪声标准差的完整计算公式:
\begin{equation}
\boxed{\sigma_n = \sqrt{\frac{P_r}{2(10^{\text{SNR}_{\text{dB}}/10} + 1)}}}
\end{equation}

这个公式在GPR去噪算法中用于准确估计噪声水平,是整个自适应去噪方法的理论基础。

\end{document}
